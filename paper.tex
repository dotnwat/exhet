% This is "sig-alternate.tex" V2.1 April 2013
% This file should be compiled with V2.5 of "sig-alternate.cls" May 2012
%
% This example file demonstrates the use of the 'sig-alternate.cls'
% V2.5 LaTeX2e document class file. It is for those submitting
% articles to ACM Conference Proceedings WHO DO NOT WISH TO
% STRICTLY ADHERE TO THE SIGS (PUBS-BOARD-ENDORSED) STYLE.
% The 'sig-alternate.cls' file will produce a similar-looking,
% albeit, 'tighter' paper resulting in, invariably, fewer pages.
%
% ----------------------------------------------------------------------------------------------------------------
% This .tex file (and associated .cls V2.5) produces:
%       1) The Permission Statement
%       2) The Conference (location) Info information
%       3) The Copyright Line with ACM data
%       4) NO page numbers
%
% as against the acm_proc_article-sp.cls file which
% DOES NOT produce 1) thru' 3) above.
%
% Using 'sig-alternate.cls' you have control, however, from within
% the source .tex file, over both the CopyrightYear
% (defaulted to 200X) and the ACM Copyright Data
% (defaulted to X-XXXXX-XX-X/XX/XX).
% e.g.
% \CopyrightYear{2007} will cause 2007 to appear in the copyright line.
% \crdata{0-12345-67-8/90/12} will cause 0-12345-67-8/90/12 to appear in the copyright line.
%
% ---------------------------------------------------------------------------------------------------------------
% This .tex source is an example which *does* use
% the .bib file (from which the .bbl file % is produced).
% REMEMBER HOWEVER: After having produced the .bbl file,
% and prior to final submission, you *NEED* to 'insert'
% your .bbl file into your source .tex file so as to provide
% ONE 'self-contained' source file.
%
% ================= IF YOU HAVE QUESTIONS =======================
% Questions regarding the SIGS styles, SIGS policies and
% procedures, Conferences etc. should be sent to
% Adrienne Griscti (griscti@acm.org)
%
% Technical questions _only_ to
% Gerald Murray (murray@hq.acm.org)
% ===============================================================
%
% For tracking purposes - this is V2.0 - May 2012

\documentclass{sig-alternate-05-2015}


\begin{document}

% Copyright
%\setcopyright{acmcopyright}
%\setcopyright{acmlicensed}
%\setcopyright{rightsretained}
%\setcopyright{usgov}
%\setcopyright{usgovmixed}
%\setcopyright{cagov}
%\setcopyright{cagovmixed}

\toappear{}
% DOI
%\doi{10.475/123_4}

% ISBN
%\isbn{123-4567-24-567/08/06}

%Conference
%\conferenceinfo{PLDI '13}{June 16--19, 2013, Seattle, WA, USA}

%\acmPrice{\$15.00}

%
% --- Author Metadata here ---
%\conferenceinfo{WOODSTOCK}{'97 El Paso, Texas USA}
%\CopyrightYear{2007} % Allows default copyright year (20XX) to be over-ridden - IF NEED BE.
%\crdata{0-12345-67-8/90/01}  % Allows default copyright data (0-89791-88-6/97/05) to be over-ridden - IF NEED BE.
% --- End of Author Metadata ---

\title{Alternate {\ttlit ACM} SIG Proceedings Paper in LaTeX Format}
%
% You need the command \numberofauthors to handle the 'placement
% and alignment' of the authors beneath the title.
%
% For aesthetic reasons, we recommend 'three authors at a time'
% i.e. three 'name/affiliation blocks' be placed beneath the title.
%
% NOTE: You are NOT restricted in how many 'rows' of
% "name/affiliations" may appear. We just ask that you restrict
% the number of 'columns' to three.
%
% Because of the available 'opening page real-estate'
% we ask you to refrain from putting more than six authors
% (two rows with three columns) beneath the article title.
% More than six makes the first-page appear very cluttered indeed.
%
% Use the \alignauthor commands to handle the names
% and affiliations for an 'aesthetic maximum' of six authors.
% Add names, affiliations, addresses for
% the seventh etc. author(s) as the argument for the
% \additionalauthors command.
% These 'additional authors' will be output/set for you
% without further effort on your part as the last section in
% the body of your article BEFORE References or any Appendices.

\numberofauthors{1} %  in this sample file, there are a *total*
% of EIGHT authors. SIX appear on the 'first-page' (for formatting
% reasons) and the remaining two appear in the \additionalauthors section.
%
\author{
% You can go ahead and credit any number of authors here,
% e.g. one 'row of three' or two rows (consisting of one row of three
% and a second row of one, two or three).
%
% The command \alignauthor (no curly braces needed) should
% precede each author name, affiliation/snail-mail address and
% e-mail address. Additionally, tag each line of
% affiliation/address with \affaddr, and tag the
% e-mail address with \email.
%
% 1st. author
\alignauthor
Noah Watkins, Michael Sevilla, Jeff LeFevre, Carlos Maltzahn
}

% There's nothing stopping you putting the seventh, eighth, etc.
% author on the opening page (as the 'third row') but we ask,
% for aesthetic reasons that you place these 'additional authors'
% in the \additional authors block, viz.

\date{30 July 1999}
% Just remember to make sure that the TOTAL number of authors
% is the number that will appear on the first page PLUS the
% number that will appear in the \additionalauthors section.

\maketitle

\section{Key challenges}

Future extreme heterogeneous systems will be managed by runtimes that handle the
complexities of abstracting across systems with diverse sets of compute
elements, deep memory hierarchies, and asymmetrical networks.  The increased
role of runtimes in resource and execution management is a direct result of the
complexity involved in designing applications that are able to portably extract
performance from complex architectures.  Often the target of domain-specific
languages, such runtimes exploit application-specific semantics and use the
state of the current system (e.g., hardware, software, configuration, etc.) to
optimize all aspects of distributed execution. Runtimes such as Legion~\cite{bauer:sc12-legion}
expect to manage the entire compute infrastructure, but real-world constraints
require access to mass storage to be shared since it is used across applications
within organizations to relieve memory pressure, archive results, and provide
critical fault-tolerant capabilities. As a result, it is unlikely that in the
near-term the entire storage system hierarchy will be managed by a single
application runtime. Thus, the ability of application runtimes to optimize
execution is partly limited by the capabilities of I/O interfaces.

\begin{quote}
  \emph{Challenge}. Current storage architectures are not designed for extreme
  heterogeneity, especially in the context of persistent memories whose
  performance is limited code path length and shared-memory designs.
\end{quote}

Current I/O interfaces are treated like black boxes by runtime systems, limited
in use to basic read-write APIs (e.g., POSIX file I/O or middleware such as
HDF5). However, such interfaces are primarily designed for consumption by human
developers, rather than runtime systems.  In order to realize end-to-end
optimizations, I/O interfaces must be enhanced to expose storage capabilities
such as fine-grained control over data durability, horizontal and vertical data
locality, and semantics that affect locking and serialization. Runtime systems
may take advantage of these capabitilies in diverse, application-specific ways.
But as a shared resource, storage systems cannot expose exclusive control to an
application runtime. In fact, important optimizations in storage systems depend
on managing access by understanding behavior across application boundaries.

\begin{quote}
  \emph{Challenge}. The design of adaptive storage I/O interfaces that safely
  provide two-way communication of with application runtimes the nature of
  workflows, hardware, data, and access semantics. Highly dynamic interfaces to shared resources
  present many optimization opporunities, but introduce difficult challenges for
  managing access.
\end{quote}

\section{Research directions}

\subsection{Interface design}

Current I/O interfaces are in large part treated as black boxes that offer
limited control in the form of \emph{magic numbers} or coarse-grain hints. With
few exceptions such as I/O sequentiality and alignment parameters, existing
interfaces allow applications to express intent or expose low-level system state
or configurability. In order to realize end-to-end application I/O optmizations,
both application runtime and storage system will need to be co-designed such that runtime
adapation of behavior based on data structure is possible.

For instance, HPC applications that use pattern-based data models and want to
optimize for checkpoint latency should be able to reliqunish control over low-level
data placement to the storage system, and in turn be able to adapatively provide
subsets of checkpoint data based on the current state of the storage system.

Historically the primary I/O interfaces have been object, block, and POSIX file
I/O. However, the availability of high-performance open-source storage systems
are allowing exploration of new storage interfaces without the fear of vendor
lock-in.  Recent work has explored the adaptability of existing storage systems
to support a range of new I/O interfaces, but an open question is how such
interfaces can efficently and reliability be constructed. Integrate KV-Drive and
LDFI into discussion.

\subsection{System architecture}

Both HPC and cloud-based systems are seeing rapid advancements towards
heterogeneity. For instance, Amazon Web Services offers a large array of node
and network capabilities including FPGA and GPU co-processors. Emerging storage
technologies such as NVMe and persistent memories are offered by cloud providers
and will be integrated in future HPC systems, but even existing storage systems
are struggling to exploit their full performance~\cite{xu:fast16-nova}. In fact, the increased
performance of next-generation media and networks suggests that future systems
may require a CPU per storage device~\cite{samuels:oss16}. And in order to support a
dynamic range of interfaces, internal sub-systems will need to be generic and
re-usable to support the diverse of demands of runtime systems.

Achieving end-to-end performance optimization will require new techniques for
verticaly integrating across application, runtime, and storage systems. The
architectural challenges of achieving low-latency I/O for persistent memories
requirements are significant. These media benefit from shared-nothing designs
that minimize context switches and \emph{cycles}, thus one-size-fits-all code
paths will be replaced by many specialized paths, potentially constructed
dynamically, that each must still support correct consistency semantics across
interfaces.

The incredible advancement of container and virtualization technologies,
dependence on domain-specific languages, and reliance on open-source software
for critical support libraries, are increasing the demand for polyglot
deployments. However, technologies for cross-language communication offer
notoriously poor performance (e.g. JNI) and will introduce unacceptable
overheads for systems with next-generation memories. New runtime systems that
rely on advanced JIT technologies are successfully being used to dynamically
perform end-to-end code path optimization across language boundaries (Graal,
Truffle, SQLiteJIT).

\subsection{Operational policies}

The data-intensive nature of both cloud and HPC applications will put extreme
pressure on shared storage systems. Even in the most constrained cases today,
efficently sharing systems is a significant challenge. Handling access to a
shared storage systems supporting a wide range of dynamic interfaces, code
paths, and new architectures that favor less shared state will require
rethinking how resource management is handled.

%
% The following two commands are all you need in the
% initial runs of your .tex file to
% produce the bibliography for the citations in your paper.
\bibliographystyle{abbrv}
\bibliography{sigproc}  % sigproc.bib is the name of the Bibliography in this case

\end{document}
